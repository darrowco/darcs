% Darcs-ED.tex
\begin{hcarentry}[updated]{Darcs}
\label{darcs}
\report{Eric Kow}%05/13
\participants{darcs-users list}
\status{active development}
\makeheader

Darcs is a distributed revision control system written in Haskell. In
Darcs, every copy of your source code is a full repository, which allows for
full operation in a disconnected environment, and also allows anyone with
read access to a Darcs repository to easily create their own branch and
modify it with the full power of Darcs' revision control. Darcs is based on
an underlying theory of patches, which allows for safe reordering and
merging of patches even in complex scenarios. For all its power, Darcs
remains a very easy to use tool for every day use because it follows the
principle of keeping simple things simple.

After three years of development, we have released Darcs 2.10 (April
2015). This new major release includes the new \verb!darcs rebase!
command (for merging and amending patches that would be hard to do with
patch theory alone), numerous optimisations and performance
improvements, a \verb!darcs convert! command for switching to and from
Git, as well as general improvements to the user interface.

\paragraph{SFC and donations}

Darcs is free software licensed under the GNU GPL (version 2 or
greater).  Darcs is a proud
member of the Software Freedom Conservancy, a US tax-exempt 501(c)(3)
organization.  We accept donations at
\url{http://darcs.net/donations.html}.

\FurtherReading
\begin{itemize}
\item \url{http://darcs.net}
\item \url{http://darcs.net/Releases/2.10}
\end{itemize}
\end{hcarentry}
